%% The following is a directive for TeXShop to indicate the main file
%%!TEX root = diss.tex

\chapter{Abstract}

In next-generation neutrino experiments such as Hyper-Kamiokande and DUNE, one of the leading sources of uncertainties lies in the calculations of neutrino flux.
 \ac{EMPHATIC} aims to improve calculations of neutrino flux by measuring hadron production resulting from a secondary hadron beam hitting a variety of targets.
\ac{EMPHATIC} includes an Aerogel Ring Imaging Cherenkov detector, which uses the distribution of detected photons resulting from Cherenkov radiation in order to determine particle velocity.
A Monte Carlo simulation is created of particles travelling through the detector, and the propagation of the generated Cherenkov radiation is simulated.
A likelihood approach is used to compare the simulated distributions to experimental measurements in order to identify particles.
The technique was found to be effective at distinguishing between protons, pions, and kaons at momenta below 7 GeV/c.
This technique was generalized to account for multi-particle events, but further work is necessary to characterize these events and evaluate the method's effiectiveness over them.

% Consider placing version information if you circulate multiple drafts
\vfill
\begin{center}
\begin{sf}
\fbox{Revision: \today}
\end{sf}
\end{center}
