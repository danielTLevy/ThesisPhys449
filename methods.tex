%% The following is a directive for TeXShop to indicate the main file
%%!TEX root = diss.tex

\chapter{Methods}
\label{ch:Methods}
The simulation of the ARICH detector was created using \textsc{ROOT}: an object-oriented scientific computing framework based off C++ and developed at CERN \cite{root}.
\textsc{ROOT} is used for its ease of use, the high degree of organization and extensibility that comes from its object-oriented nature, and its high efficiency when compiled.
The simulation was created in steps of increasing added complexity, which are outlined in this chapter.


\section{Initial Simulation}
\label{sec:experiment}
The first step in building the simulation was to generate a stream of particles, representing the charged pions and kaons and protons produced in an experiment.
If we define the $z$ axis to be the downstream direction, then the inputs to the simulation are:
\begin{itemize}
\item the initial $x$ and $y$ position of a particle at $z=0$,
\item the error on the $x$ and $y$ position of the particle, corresponding to the position resolution of the upstream particle tracker,
\item the $x$ and $y$ components of the unit vector representing the direction of the particle's movement,
\item the error on the $x$ and $y$ direction of the particle, corresponding to the direction resolution of the upstream particle tracker, and
\item the velocity $\beta$ of the particle.
\end{itemize} 
The simulation generates 10,000 such particles whose initial positions and directions are randomly drawn from a Gaussian distribution with the specified mean values and errors. 

Initially, the aerogel was defined as a 2 cm thick, 10 cm by 10 cm volume, with a refractive index of 1.035. 
For a given particle velocity $\beta$, we calculate the mean number of photons per unit length generated by a charged particle passing through with equation \ref{eq:photonNumber}. For each particle, we 

For each generated particle, we generate a number of photons equal to the result of equation \ref{eq:photonNumber} multiplied by the distance the particle has to travel through the aerogel.
The initial position of these photons are randomly distributed along the path travelled by the particle.
Their polar angle $\theta$ with respect to the direction of travel of the particle is given by equation \ref{eq:cherenkovAngle}, and their azimuthal angle is randomly drawn from between 0 and $2\pi$. Given 
If we are given the direction vector of the particle, we can use a rotation matrix to get the resulting direction vectors for each of the photons it generates.
This is given by \TODO{Include here Rodrigue's rotation formula, and how it is adapted into matrix form}.

The photons ultimately are detected by an array of PMTs, which have a known quantum efficiency, characterized over a range of frequencies of light.
The efficiency curve of the PMTs is plotted in \TODO{cite datasheet?} Figure \TODO{INCLUDE QUANTUM EFF GRAPH}, and is given over a range from 267 to 687 nm.
From equation \ref{eq:frankTamm}, we see that the distribution of photon wavelengths follows the distribution $\frac{1}{\lambda^2}$, so each photon is produced with a wavelength randomly drawn from this distribution on a range from 250 nm to 695 nm.
These values were chosen, because if we linearly extrapolate the quantum efficiency curve, we see that we do not expect to get any detector response for photons of wavelengths outside of this range.
Because all optical processes included in this simulation are inelastic, we do not expect any photon wavelength to change between the point of generation and the point of detection.
Therefore, after generating Cherenkov photons we may randomly reject them immediately based on their probability of being detected, and not simulate their paths any further.

The PMTs have dimensions of 48.5 mm $\times$ 48.5 mm.
Each PMT contains $8 \times 8$ pixels, and have a packing density of 87\%. 
The PMTs are expected to be arranged onto a 30 cm $\times$ 30 cm plane with a packing efficiency of $80\%$, 20 cm downstream of the first layer of aerogel.
\TODO{Find out where packing efficiency of 0.8 comes from}.
To model the photon detector array, we simply project the photons downstream onto a 30 cm $\times$ 30 cm histogram with $48 \times 48$ bins.
The number of photons counted in each bin is scaled by a factor of $80\% \times 87\%$ to account for the packing efficiencies.

Two examples of photon distribution arising from this simple simulation are shown in Figure \TODO{Include the two distributions from the presentation, but with better titles perhaps}.
While this simulation well approximates the expected location of a particle's photon rings for a single layer of aerogel, it is missing a number of key effects that would allow it to actually be used to calculate a realistic photon distribution.

\section{Optical Effects}
Following this initial simulation, more sophisticated optical effects were added in. 

\subsection{Aerogel Choice}


$d1*tan(acos(1.0/n1)) = d2*tan(acos(1.0/n2)) $

\subsection{Rayleigh Scattering}

For EMPHATIC, aerogels with indices of refraction ranging from 1.03 to 1.05 in increments of 0.005 were considered, each with a thickness of 2 cm.
The properties of different typical aerogels were characterized by another member of the collaboration \TODO{give credit to Tabata}: their refractive indices were measured for light with a wavelength of 405 nm, and the transmittances of light were measured for light of wavelengths ranging from 190 nm to 800 nm.
The transmittance and refractive index of each aerogel are shown in Figure \TODO{aerogel refractive indices}.

As described in Section \ref{sec:optics}, the primary optical effect to account for is Rayleigh Scattering, wherein photons scatter with a probability proportional to $\lambda^{-4}$, and in a direction proportional to $1 + \cos^2(\theta)$.
If we assume that the transmittance of light through the aerogels is only affected by Rayleigh scattering then we can approximate $L(\lambda)$, the Rayleigh scattering interaction length at each wavelength, with the following formula:

\begin{equation}
L(\lambda) = \frac{-d}{\log(T(\lambda))}
    \label{eq:scatLength}
\end{equation}

In this equation, $T(\lambda)$ is the transmittance of light at a wavelength $\lambda$ across some thickness $d$ of the aerogel.
In order to efficiently sample a random distance before scattering, we can apply inverse transform sampling \TODO{Cite inverse transform sampling}.
The probability $p$ that a photon has scattered after travelling some distance $x$ through the aerogel is given by

\begin{equation}
p = 1 - \exp(-x/L)
    \label{eq:scatProb}
\end{equation}

To randomly draw a distance $x$ from this distribution, we can sample some $p$ uniformly on the range $[0,1]$, and get a distance $x$ by calculating the inverse:

\begin{equation}
x =   -L\ln(1-p)
  \label{eq:randomScat}
\end{equation}

In the simulation, when each photon is generated, a random ``distance until scattering" is calculated this way. 
We check whether or not a photon would still be in the aerogel after having travelled that distance: if it is, then we advance it forwards by that amount, and then assign the photon a new direction, where $\theta$ is sampled from the distribution $1 + \cos^2(\theta)$, and $\phi$ is randomly sampled from $[0, 2\pi]$.
After the position and direction of the photon has been updated, a new ``distance until scattering" is calculated and this process is repeated. 
If we determine that a photon would not be in the aerogel after having travelled that distance, then we advance the photon forwards until the boundary of the aerogel.
We then check what material the photon would be entering, and update the direction of the photon to account for the refraction caused by the photon travelling between two mediums of different refractive indices.


An example of a photon distribution is shown in Figure \ref{fig:photonHist}.

\begin{figure}[]
\centering
\resizebox{0.9\textwidth}{!}{\includegraphics{./figs/photonHist.pdf}}
\caption[Example of simulated photon distribution for centered 7.0 GeV pion beam]{Histograms displaying the simulated photon distribution for a 7.0 GeV pion beam travelling directly along the $z$-axis. Clockwise, from bottom left: (1) histogram of mean detected photon count per pixel on detector plane, (2) $y$-axis projection of photon distribution, (3) histogram of photon detections as function of distance from origin, (4) $x$-axis projection of photon distribution. }
\label{fig:photonHist} 
\end{figure}

\subsection{Refraction}


\section{Particle Identification}
\label{sec:particleIdentification}
In order to identify particles using the \ac{ARICH} detector, a particle likelihood method is used \cite{richImpact, belleArich}.
For this method, we take the measured momentum of the particle, and use Equation \TODO{Include relativistic equation for mass / momentum / velocity here } to determine the expected velocity for different candidate particles masses.
The candidate particles of interest for this study are protons, pions, and kaons.
For each velocity hypothesis, we run 10,000 simulations of particles moving at that velocity, with the same measured initial trajectory as input, and get the distribution of the resulting photons.
For each pixel of the detector, this procedure will give a value $\lambda_i(\beta)$, equal to the expected number of photons striking pixel $i$ in the detector due to a particle of velocity $\beta$. 

For a given experimental event, we will detect $N_i$ photons in each pixel $i$.
In reality, the PMTs are only capable of registering whether or not a photon has been detected - if multiple photons strike a single pixel in a very short amount of time, it will not be able to distinguish the number detected, so we just know if $N_i = 0$ or $N_i > 0$.

If we assume that the number of photons striking detectable by a PMT pixel is given by a Poisson distribution, then the probability that zero photons strike pixel $i$ is given by:
$$ P_i(N_i=0; \beta) = e^{-\lambda_i(\beta)} $$
 The probability that one or more photons strike pixel $i$ must then be:
$$ P_i(N_i>1; \beta) = 1 - e^{-\lambda_i(\beta)} $$

By multiplying the probabilities of getting the observed result in each pixel $i$ of the detector, we calculate the likelihood for that value of $\beta$:

$$L_\beta = \prod_{i}P_i(N_i; \beta)$$

Because the log function is montonic, instead of maximizing the likelihood function, we can instead minimize the negative log-likelihood, which is easier to compute:
\begin{equation}
    \label{eq:loglikelihood}
    -2\ln(L_\beta) = -2\sum_i \ln(P_i(N_i; \beta))
\end{equation}

\TODO{Due to the relatively high error associated with the measurement of the particle momenta, if I find that this method is not entirely sufficient at identifying particles then I will enhance the technique by scanning over different momentum hypotheses. This has not yet been implemented}

\section{Multi-particle events}
\TODO{In this section, I will  talk about multi-particle events. It is often the case that the photons from several different particles will be detected at the same time. The resulting photon distributions cannot necessarily be disentangled, so I will have to look at how to fit both particles at once. This has not yet been implemented.}


\endinput

Any text after an \endinput is ignored.
You could put scraps here or things in progress.
