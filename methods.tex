%% The following is a directive for TeXShop to indicate the main file
%%!TEX root = diss.tex

\chapter{Methods}
\label{ch:Methods}
\TODO{Include some pretty figures of the photon distributions I get at each step of building the simulation}

The simulation of the ARICH detector was created using \textsc{ROOT}: an object-oriented scientific computing framework based off C++ and developed at CERN \cite{root}.
\textsc{ROOT} is used for its ease of use, the high degree of organization and extensibility that comes from its object-oriented nature, and its high efficiency when compiled.
\textsc{PROOF} is an extension to \textsc{ROOT} that allows for parallelization of the code. The simulation was created in steps of increasing added complexity, which are outlined in this chapter.


\section{Initial Simulation}
\label{sec:experiment}
The first step in building the simulation was to generate particles, and have each of them generate photons at the appropriate angle as they travel through a single layer of aerogel. If we define the $z$ axis to be the downstream direction, then the inputs to the simulation are:
\begin{itemize}
\item The $x$ and $y$ position of a particle as it enters the aerogel.
\item The error on the $x$ and $y$ position of the particle, corresponding to the position resolution of the upstream particle tracker.
\item The $x$ and $y$ components of the unit vector representing the direction of the particle.
\item The error on the $x$ and $y$ direction of the particle, corresponding to the direction resolution of the upstream particle tracker.
\item The velocity $\beta$ of the particle.
\end{itemize} 
We generate 10,000 such particles whose positions and directions are randomly drawn from a Gaussian distribution with the specified mean values and errors. 
Each of these particles generates a number of photons equal to the result of equation \ref{eq:photonNumber} multiplied by the distance the particle has to travel through the aerogel.
These photons are randomly distributed along the path travelled by the particle.
Their polar angle $\theta$ with respect to the direction of travel of the particle is given by equation \ref{eq:cherenkovAngle}, and their azimuthal angle is randomly drawn from between 0 and $2\pi$. 
If we are given the direction vector of the particle, we can use a rotation matrix to get the resulting direction vectors for each of the photons it generates.
This is given by \TODO{Include here Rodrigue's rotation formula, and how it is adapted into matrix form}.
Each photon is generated with a corresponding wavelength, drawn randomly from \TODO{Include here the integrated form of the Frank Tamm formula}. 

Given the point of generation and the direction vector of each photon, we can then determine where on the PMT array the photon will hit.
In order to approximate the pixels on the PMT array, we represent the detector as a 30cm by 30cm plane, binned into a 48 by 48 array of segments.
To account for the fill efficiency of the PMTs in the detector, we scale the number of photons hits in each bin first by $80\%$, and then by $87\%$.
We also know the quantum efficiency of the PMTs across different photon wavelengths.
To account for this and reduce computation time, as soon as we generate a photon, we look at the detector efficiency at that wavelength. 
We generate a random number from 0 to 1, and compare to this quantum efficiency - if the random number if higher than the efficiency, we throw out the photon and do not simulate it further.

\section{Optical Effects}
Following this initial simulation, more sophisticated optical effects were added in.
As described in Section \ref{sec:optics}, the primary optical effect to account for is Rayleigh Scattering, wherein photons scatter with a probability proportional to $\lambda^{-4}$, and in a direction proportional to $1 + \cos^2(\theta)$.
For each aerogel available for use in the experiment, the transmittance had been measured at different wavelengths of light.
For each wavelength, the Rayleigh scattering interaction length was approximated by the following formula:

$$
L(\lambda) = \frac{-d}{\log(T(\lambda))}
 $$
 In this equation, $T(\lambda)$ is the transmittance of light at a wavelength $\lambda$, $d$ is the thickness of the aerogel, and $L$ is the estimated mean interaction length of a photon before undergoing Rayleigh scattering.

\TODO{Here I will: \\
- go more in-depth of how I efficiently determine at what step a photon will scatter \\
- talk about multiple-scattering \\
- talk about refraction
}

An example of a photon distribution is shown in Figure \ref{fig:photonHist}.

\begin{figure}[]
\centering
\resizebox{0.9\textwidth}{!}{\includegraphics{./figs/photonHist.pdf}}
\caption[Example of simulated photon distribution for centered 7.0 GeV pion beam]{Histograms displaying the simulated photon distribution for a 7.0 GeV pion beam travelling directly along the $z$-axis. Clockwise, from bottom left: (1) histogram of mean detected photon count per pixel on detector plane, (2) $y$-axis projection of photon distribution, (3) histogram of photon detections as function of distance from origin, (4) $x$-axis projection of photon distribution. }
\label{fig:photonHist} 
\end{figure}

\section{Particle Identification}
\label{sec:particleIdentification}
In order to identify particles using the \ac{ARICH} detector, a particle likelihood method is used \cite{richImpact, belleArich}.
For this method, we take the measured momentum of the particle, and use Equation \TODO{Include relativistic equation for mass / momentum / velocity here } to determine the expected velocity for different candidate particles masses.
The candidate particles of interest to the experiment are protons, electrons, pions, and kaons.
For each velocity hypothesis, we run 10,000 simulations of particles moving at that velocity, with the same measured initial trajectory as input, and get the distribution of the resulting photons.
For each pixel of the detector, this procedure will give a value $\lambda_i(\beta)$, equal to the expected number of photons striking pixel $i$ in the detector due to a particle of velocity $\beta$. 

For a given experimental event, we will detect $N_i$ photons in each pixel $i$.
In reality, the PMTs are only capable of registering whether or not a photon has been detected - if multiple photons strike a single pixel in a very short amount of time, it will not be able to distinguish the number detected, so we just know if $N_i = 0$ or $N_i > 0$.

The probability that zero photons strike pixel $i$ is given by the Poisson distribution for zero events:
$$ P_i(N_i=0; \beta) = e^{-\lambda_i(\beta)} $$
 The probability that one or more photons strike pixel $i$ must then be:
$$ P_i(N_i>1; \beta) = 1 - e^{-\lambda_i(\beta)} $$

By multiplying the probabilities of getting the observed result in each pixel $i$ of the detector, we calculate the likelihood for that value of $\beta$:

$$L_\beta = \prod_{i}P_i(N_i; \beta)$$

For convenience, we actually compute:
\begin{equation}
    \label{eq:loglikelihood}
    -2\ln(L_\beta) = -2\sum_i \ln(P_i(N_i; \beta))
\end{equation}

We compute the log-likelihood of our data matching each of particle hypothesis, and choose that which minimizes the value.

\TODO{Due to the relatively high error associated with the measurement of the particle momenta, if I find that this method is not entirely sufficient at identifying particles then I will enhance the technique by scanning over different momentum hypotheses. This has not yet been implemented}

\section{Multi-particle events}
\TODO{In this section, I will  talk about multi-particle events. It is often the case that the photons from several different particles will be detected at the same time. The resulting photon distributions cannot necessarily be disentangled, so I will have to look at how to fit both particles at once. This has not yet been implemented.}


\endinput

Any text after an \endinput is ignored.
You could put scraps here or things in progress.
